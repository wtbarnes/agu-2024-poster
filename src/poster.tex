% Gemini theme
% https://github.com/anishathalye/gemini

\documentclass[final]{beamer}

% ====================
% Packages
% ====================

\usepackage[T1]{fontenc}
\usepackage{lmodern}
% Lengths are in centimeters: size is 4 ft by 3 ft
\usepackage[size=custom, height=91.44, width=121.92, orientation=landscape, scale=1.0]{beamerposter}
\usetheme{gemini}
\usecolortheme{wtbarnes}
\usepackage{graphicx}
\usepackage{booktabs}
\usepackage{tikz}
\usepackage{pgfplots}
\pgfplotsset{compat=1.14}
\usepackage{anyfontsize}
\usepackage{multicol}
\usepackage[authoryear]{natbib}
\usepackage{import}
\usepackage{siunitx}
\usepackage{float}

% ====================
% Lengths
% ====================

% If you have N columns, choose \sepwidth and \colwidth such that
% (N+1)*\sepwidth + N*\colwidth = \paperwidth
\newlength{\sepwidth}
\newlength{\colwidth}
\setlength{\sepwidth}{0.01\paperwidth}
\setlength{\colwidth}{0.32\paperwidth}
\newcommand{\separatorcolumn}{\begin{column}{\sepwidth}\end{column}}
% This is to work around a bug in matplotlib that leaves a command undefined when
% using pgf figures: https://github.com/matplotlib/matplotlib/issues/27907
\def\mathdefault#1{\displaystyle #1}

% ====================
% Title
% ====================
\title{How Does Heating Frequency Vary with Active Region Age?}
\author{
  W. T. Barnes \inst{1}\textsuperscript{,}\inst{2} \and
  S. J. Bradshaw \inst{3} \and
  N. M. Viall \inst{2} \and
  E. M. Mason \inst{4}
}
\institute[]{
  \inst{1} Department of Physics, American University \samelineand
  \inst{2} Heliophysics Science Division, NASA Goddard Space Flight Center \and
  \inst{3} Department of Physics and Astronomy, Rice University \samelineand
  \inst{4} Predictive Science, Inc.
}

% ====================
% Footer (optional)
% ====================
\footercontent{
  \href{https://github.com/wtbarnes/agu-2024-poster}{github.com/wtbarnes/agu-2024-poster} \hfill
  AGU24 --- SH33B: New Insights into Coronal Physics from EUV and UV Spectroscopy --- 11 December 2024 \hfill
  \href{mailto:wbarnes@american.edu}{wbarnes@american.edu}
}

% ====================
% Logo (optional)
% ====================
\logoleft{\includegraphics[height=8cm]{static/agu24_logo.png}}
\logoright{\includegraphics[height=10cm]{static/sunpy_logo_portrait_powered.png}}

% ====================
% Body
% ====================

\begin{document}

\begin{frame}[t]
\begin{columns}[t]
\separatorcolumn

\begin{column}{\colwidth}

  \begin{block}{Introduction}

    \begin{itemize}
      \item Heating frequency: \alert{frequency at which elemental strands in the corona are reenergized}
      \begin{itemize}
        \item High frequency (HF): waiting time between successive events less than a loop cooling time
        \item Low frequency (LF): waitign time between successive events greater than a loop cooling time
      \end{itemize}
      \item \alert{Heating frequency is variable over an active region} \citep[e.g.][]{del_zanna_evolution_2015,barnes_understanding_2021}
      \begin{itemize}
        \item Warm, periphery loops exhibit low-frequency heating \citep[e.g.][]{warren_evolving_2003}
        \item Hot, core loops more consistent with high-frequency heating \citep[e.g.][]{warren_evidence_2010}
      \end{itemize}
      \item \alert{Active region properties evolve with age:} ``very hot'' emission \citep{ugarte-urra_is_2012,ugarte-urra_determining_2014}, abundance \citep{testa_coronal_2023}, density and temperature \citep{ko_correlation_2016}
      \item \textbf{Goal:} Understand how the distribution of heating frequencies in an active region evolves with age
    \end{itemize}

  \end{block}

  \begin{block}{NOAA Active Regions 11944, 11967, and 11990}

    Nam vulputate nunc felis, non condimentum lacus porta ultrices. Nullam sed
    sagittis metus. Etiam consectetur gravida urna quis suscipit.

    Eget augue porta, bibendum venenatis tortor.

    \begin{figure}
      \centering
      \import{figures/}{aia_context.pgf}
      \caption{Context images for the active region over three rotations.} 
      \label{fig:aia_context}
    \end{figure}

  \heading{AIA Observations}

    This block catches your eye, so \textbf{important stuff} should probably go
    here.
  
    \begin{figure}
      \centering
      \import{figures/}{aia_cutouts.pgf}
      \caption{Cutouts for the active region over all three rotations in each AIA EUV channel.} 
      \label{fig:aia_cutouts}
    \end{figure}

  \heading{EIS Observations}

    \begin{itemize}
      \item \textbf{Sed consequat} id ante vel efficitur. Praesent congue massa
        sed est scelerisque, elementum mollis augue iaculis.
      \item \textbf{Sed luctus, elit sit amet} dictum maximus, diam dolor
        faucibus purus, sed lobortis justo erat id turpis.
      \item \textbf{Pellentesque facilisis dolor in leo} bibendum congue.
        Maecenas congue finibus justo, vitae eleifend urna facilisis at.
    \end{itemize}

    \begin{figure}
      \centering
      \import{figures/}{eis_rasters.pgf}
      \caption{EIS raster scans that coincide with the AR cutouts shown above.} 
      \label{fig:eis_rasters}
    \end{figure}
  
  \end{block}

\end{column}
\separatorcolumn  
\begin{column}{\colwidth}

  \begin{block}{Masking Out Flaring Regions}

    \begin{columns}[c]
      \begin{column}{0.5\colwidth}
        \begin{figure}
          \centering
          \import{figures/}{time_lag_mask.pgf}
          \caption{Examples of time lag maps and intensity map with a flare mask applied.} 
          \label{fig:flare_masked_maps}
        \end{figure}
      \end{column}
      \begin{column}{0.5\colwidth}
        \begin{enumerate}
          \item \textbf{Morbi mauris purus}, egestas at vehicula et, convallis
            accumsan orci. Orci varius natoque penatibus et magnis dis parturient
            montes, nascetur ridiculus mus.
          \item \textbf{Cras vehicula blandit urna ut maximus}. Aliquam blandit nec
            massa ac sollicitudin. Curabitur cursus, metus nec imperdiet bibendum,
            velit lectus faucibus dolor, quis gravida metus mauris gravida turpis.
        \end{enumerate}
        \begin{figure}
          \centering
          \import{figures/}{ebtel_timelag_simulation.pgf}
          \caption{\texttt{EBTEL} simulation of flare- and nanoflare-like events. The top panel shows the temperature and intensity for the 131, 193, and 211 Å channels for each event. The bottom panel shows the cross-correlation curves for two channel pairs.} 
          \label{fig:ebtel_timelag}
        \end{figure}
      \end{column}
    \end{columns}

  \end{block}

  \begin{block}{Emission Measure Distributions}

    \begin{figure}
      \centering
      \import{figures/}{em_slope_map.pgf}
      \caption{Emission measure slope in each pixel of the AR over all three rotations as determined from the DEM derived from the six AIA EUV channels.} 
      \label{fig:em_slope_map}
    \end{figure}

    \begin{figure}
      \centering
      \import{figures/}{emslope_distribution.pgf}
      \caption{Emission measure slope distributions for the AIA FOV (left) and the EIS FOV (right). The dashed distributions in the right panel are the slopes calculated from the DEM derived from the fitted EIS spectra.} 
      \label{fig:emslope_distribution}
    \end{figure}

  \end{block}

\end{column}

\separatorcolumn

\begin{column}{\colwidth}

  \begin{block}{Time Lag Analysis}

    \begin{figure}
      \centering
      \import{figures/}{timelag_maps.pgf}
      \caption{Time lag in each pixel of each active region for four different channel pairs.}
      \label{fig:timelag_maps}
    \end{figure}

    \begin{figure}
      \centering
      \import{figures/}{timelag_distribution.pgf}
      \caption{Distributions of time lags for four different channel pairs for the active region over all three rotations.}
      \label{fig:timelag_distribution}
    \end{figure}

  \end{block}

  \begin{block}{Conclusions}

    \begin{itemize}
      \item Identified active region across \alert{three rotations} in both Hinode/EIS and SDO/AIA
      \item Computed \alert{time lags} and \alert{emission measure slopes} across active region for each rotation
      \item Later rotations show \alert{narrower emission measure slope distribution with lower peak}
      \item \alert{More positive time lags} in cooler channels at later rotations, but \alert{more negative time lags} in pairs that include a bimodal channel with a hot component
      \item Preliminary results imply \alert{less frequent, less energetic events as active region ages}
      \item \alert{Future:} comparisons with magnetic properties, correlations with doppler shifts and widths, apply to more active regions
    \end{itemize}

  \end{block}

  \begin{block}{References}
    \scriptsize
    This research is funded by NASA grant 80NSSC24K0256 through the ROSES 2023 Heliophysics Supporting Research program.
    This research used v6.0.4 of \texttt{sunpy} \citep{the_sunpy_community_sunpy_2020} and v0.9.1 of \texttt{aiapy} \citep{barnes_aiapy_2020}, and v0.97.1 of \texttt{eispac} \citep{weberg_eispac_2023}.
    \begin{multicols}{2}
      \bibliographystyle{aasjournal.bst}
      \bibliography{references.bib}
    \end{multicols}
  \end{block}

\end{column}

\separatorcolumn
\end{columns}
\end{frame}

\end{document}
